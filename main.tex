\documentclass[oneside,final,14pt]{extreport}

% Изменяем шрифт
\usepackage{fontspec}
\setmainfont{Times New Roman}
\listfiles

\usepackage{mathtools}

\usepackage[russianb]{babel}


% Работа с отступами
\usepackage{vmargin}
\setpapersize{A4}
% отступы
%\setmarginsrb 
%{3cm} % левый
%{2cm} % верхний
%{1cm} % Правый
%{2cm} % Нижний
%{0pt}{0mm} % Высота - отступ верхнего колонтитула
%{0pt}{0mm} % Высота - отступ нижнего  колонтитула
\usepackage{indentfirst}

% Убиваем перенос
\usepackage[none]{hyphenat}

% выравнивание по ширине
\usepackage{microtype}

\setlength\hoffset{0cm}
\setlength\voffset{0cm}
\usepackage[top=2cm, bottom=2cm, left=3cm, right=2cm,
]{geometry}



\usepackage{titlesec}



% Настройка  заглавий

\titleformat
{\chapter} % command
[display]
{
\bfseries
} % format
{
\thechapter.
} 	% label
{ 
	0 pt
} % sep
{    
\centering
} % before-code

% Конец настройка заглавий


% веелечина отступа
%\setlength\parindent{0pt}



\sloppy
\usepackage{layout}

\begin{document}

\tableofcontents
\newpage
\chapter*{Введение}

В Сибирском государственном университете телекоммуникаций и информатики,с которым у КубГУ имеется соглашение о сотрудничестве в сфере образования, науки, научной и инновационной деятельности стоит задача оценивания
качества контактной работы, реализуемой посредством вебинаров, в ходе дистанционного обучения. 
Для решения этой задачи разработана математическая модель, которая включает систему из 32 показателей качества организации и проведения вебинара (перечень которых представлен в приложении), позволяющих оценить с разных сторон  качество вебинара 10-балльными экспертными оценками. Проблема заключается в существенных трудозатратах, которые несут эксперты при оценивании этих показателей. Кроме того, мнения экспертов субъективны, а задача поставлена максимально объективизировать процедуру оценивания, например, за счет минимизации влияния человеческого фактора в процедуре оценивания. В связи с этим актуальной является  задача разработки такой компьютерной технологии, которая позволит оценить максимальное количество показателей без участия человека в автоматическом режиме. В числе показателей, которые могут быть автоматически при помощи некоторого алгоритма входят: колличество и качество информации на слайдах, использование указки, точность заявленной длительности мероприяти и тд.

Основой для анализа данных показателей является три основных аспекта:

поиск указки

выделение слайдов

Распознавание блоков текста на слайдах

	Для решения данных задач в ходе курсовой работы были рассмотрены алгоритмы поиска границ обектов на изображении, сравнения двух изображений, поиска шаблона в изображении, а также методы локализации текста среди прочих объектов. Кроме того, изучены методы, которые позволяют значительно ускорить работу всех вышеперечисленных алгоритмов. Програмно было реализованны процедуры, реализующие основу для дальнейшего развития приложения автоматического вычисления некоторых показателей вебинаров. 

	Для решения данных задач в ходе курсовой работы были рассмотрены алгоритмы поиска границ обектов на изображении, сравнения двух изображений, поиска шаблона в изображении, а также методы локализации текста среди прочих объектов. Кроме того, изучены методы, которые позволяют значительно ускорить работу всех вышеперечисленных алгоритмов. Програмно было реализованны процедуры, реализующие основу для дальнейшего развития приложения автоматического вычисления некоторых показателей вебинаров. 


	Для решения данных задач в ходе курсовой работы были рассмотрены алгоритмы поиска границ обектов на изображении, сравнения двух изображений, поиска шаблона в изображении, а также методы локализации текста среди прочих объектов. Кроме того, изучены методы, которые позволяют значительно ускорить работу всех вышеперечисленных алгоритмов. Програмно было реализованны процедуры, реализующие основу для дальнейшего развития приложения автоматического вычисления некоторых показателей вебинаров. 


	Для решения данных задач в ходе курсовой работы были рассмотрены алгоритмы поиска границ обектов на изображении, сравнения двух изображений, поиска шаблона в изображении, а также методы локализации текста среди прочих объектов. Кроме того, изучены методы, которые позволяют значительно ускорить работу всех вышеперечисленных алгоритмов. Програмно было реализованны процедуры, реализующие основу для дальнейшего развития приложения автоматического вычисления некоторых показателей вебинаров. 


	Для решения данных задач в ходе курсовой работы были рассмотрены алгоритмы поиска границ обектов на изображении, сравнения двух изображений, поиска шаблона в изображении, а также методы локализации текста среди прочих объектов. Кроме того, изучены методы, которые позволяют значительно ускорить работу всех вышеперечисленных алгоритмов. Програмно было реализованны процедуры, реализующие основу для дальнейшего развития приложения автоматического вычисления некоторых показателей вебинаров. 


	Для решения данных задач в ходе курсовой работы были рассмотрены алгоритмы поиска границ обектов на изображении, сравнения двух изображений, поиска шаблона в изображении, а также методы локализации текста среди прочих объектов. Кроме того, изучены методы, которые позволяют значительно ускорить работу всех вышеперечисленных алгоритмов. Програмно было реализованны процедуры, реализующие основу для дальнейшего развития приложения автоматического вычисления некоторых показателей вебинаров. 


	Для решения данных задач в ходе курсовой работы были рассмотрены алгоритмы поиска границ обектов на изображении, сравнения двух изображений, поиска шаблона в изображении, а также методы локализации текста среди прочих объектов. Кроме того, изучены методы, которые позволяют значительно ускорить работу всех вышеперечисленных алгоритмов. Програмно было реализованны процедуры, реализующие основу для дальнейшего развития приложения автоматического вычисления некоторых показателей вебинаров. 

	
\chapter{Методы определения границ объектов на изображении}

Границы объектов на изображении характеризуются изменением яркости в некотором направлении. Выделяют три вида границ: идеальные,  размытые  и крышевидные. На рис № изображены вертикальные границы трех видов.

картинки

Рассмотрим вертикальную размытую границу. Тогда зафиксировав ординату, получим дискретную функцию от одной переменной g(x). Рассмотрим её первую и вторую производные: первая производная равна нулю в областях, где интенсивность постоянна и равна константе на границе, причем константа тем больше, чем уже граница. Вторая производная не равна нулю только в координатах начала и конца границы g(x) В этих точках она равна бесконечности. Для случая крыловидной границы, первая производная положительна на подъеме и отрицательна на спуске границы. Вторая производная не ноль в трех точках, причем граница помещается между первой и последней не нулевой точкой второй производной. Отсюда следует вывод: зная направление границы, её координаты находятся из производной функции интенсивности по этому направлению. 
 
В общем случае, если заранее неизвестны направления границ объектов, за направление берут направление максимального роста интенсивности  т.е. градиент изображения, если представлять его как дискретную функцию от двух переменных. 

Градиент есть вектор частных производных, производные вычисляются численно, разностными методами. Если принять расстояние между соседними в строке и соседними в столбце пикселями за единицу, компоненты градиента с точностью O(1) вычисляется по формулам

Что равносильно свертке изображения с ядрами 

На практике для вычисления градиента используются оператор Собеля

В результате получаем чб изображение, на котором белым цветом изображены предполагаемые границы.

\end{document} 